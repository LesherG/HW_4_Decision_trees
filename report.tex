\documentclass{article}
\usepackage{amsmath}
\usepackage{graphicx}
\usepackage[margin=1in]{geometry}
\usepackage{indentfirst}

\title{HW 4}
\author{Gavin Lesher}
\date{\today}
\begin{document}
    \maketitle
    \cleardoublepage
    \section*{Q1: Drawing Decision Tree Predictions}
        \subsection*{a)}
            \begin{tabular}{c c c}
                A:[0,10) & C: \O & E: \O \\
                B:[10,15) & D: [15, 25) & F: [25,30]
            \end{tabular}
        \subsection*{b)}
        \begin{figure}[h]
            \includegraphics[width=.5\textwidth]{"Q1_b.png"}
        \end{figure}
        \subsection*{c)}
            It mekes it easier to find an accurate tree because you have more ending trees to select from. However, it does not make it easier to find an \emph{optimal} tree.


    \section*{Q2: Manualy Learning A Decision Tree}
        We must first compute the entropy of the original table.
        \begin{equation*}
            H(y) = -\frac{1}{2} log_2 \frac{1}{2} - \frac{1}{2} log_2 \frac{1}{2} = 1
        \end{equation*}
        \begin{center}
            \begin{tabular}{c | l}
                Split On & \\
                \hline
                A & $H(y) - H(y|A) \Rightarrow-\frac{1}{2}\left(\frac{2}{3}log_2 \frac{2}{3} + \frac{1}{3}log_2 \frac{1}{3}\right) -\frac{1}{2}\left(\frac{1}{3}log_2 \frac{1}{3} + \frac{2}{3}log_2 \frac{2}{3}\right)$\\
                B & $H(y) - H(y|B) \Rightarrow -\frac{1}{3}\left(\frac{1}{2}log_2 \frac{1}{2} + \frac{1}{2}log_2 \frac{1}{2}\right) -\frac{2}{3}\left(\frac{1}{2}log_2 \frac{1}{2} + \frac{1}{2}log_2 \frac{1}{2}\right)$\\
                C & $H(y) - H(y|C) \Rightarrow -\frac{1}{2}\left(\frac{2}{3}log_2 \frac{2}{3} + \frac{1}{3}log_2 \frac{1}{3}\right) -\frac{1}{2}\left(\frac{1}{3}log_2 \frac{1}{3} + \frac{2}{3}log_2 \frac{2}{3}\right)$\\
                &\\
                A & = 1 - .9183 = .0817 \\
                B & = 1 - 1 = 0\\
                C & = 1 - .9183 = .0817
            \end{tabular}   
        \end{center}
        \qquad We can split on either A or C, so we arbitrarily choose A.
        I will first work on the dataset where $A=0$ with data:
        \begin{center}
            \begin{tabular}{c | c c | c}
                A & B & C & Y \\
                \hline
                0 & 1 & 1 & 0 \\
                0 & 0 & 0 & 0 \\
                0 & 1 & 0 & 1 \\
            \end{tabular}
        \end{center}
        \qquad For simplicity I will be leaving out the intermediary calculations of $H(y | X)$
        \begin{center}
            \begin{tabular}{c | l}
                Split On & \\
                \hline
                B & $.9183 - H(y | B) = .9183 - \frac{2}{3} = .2516$\\
                C & $.9183 - H(y | C) = .9183 - \frac{2}{3} = .2516$\\
            \end{tabular}
        \end{center}
        \qquad Again, we arbitrarily choose B to split on. The dataset $A=0, B=0 $ only has one entry, so we can make a leaf node that predicts 0 for $B = 0$. 
        We then move onto our final split, but since we only have C to split on, we automatically do that, and create 2 more leaf nodes off of that split where $C = 0$ predicts 1, and $C=1$ predicts 0.
        Our tree now looks as follows: 
        \begin{figure}[h]
            \centering
            \includegraphics[width=.5\textwidth]{"Q2_1.png"}
        \end{figure}

        \qquad We now move onto the empty sub-tree of $A=1$. 
        Following the same process:
        \begin{center}
            \begin{tabular}{c | l}
                Split On & \\
                \hline
                B & $.9183 - H(y | B) = .9183 - \frac{2}{3} = .2516$\\
                C & $.9183 - H(y | C) = .9183 - \frac{2}{3} = .2516$\\
            \end{tabular}
        \end{center}
        \qquad Split on B, create leaf node with $B=0$ predicting 1, then split on C to have the final predictions of $C=1$ predicts 0 and $C=0$ predicts 1. 
        This makes our final tree have the following structure, and a training accuracy of 100\%. 
        \begin{figure}[h]
            \centering
            \includegraphics[width=.5\textwidth]{"Q2_2.png"}
        \end{figure}
        \clearpage


    \section*{Q3: Measuring Correlation in Random Forests}
        \subsection*{a)}
            \begin{figure}[h]
                \centering
                \includegraphics[width=\textwidth]{"with_bagging.png"}
            \end{figure}
        \qquad With bagging, the correlation between models goes down a good amount. For simply sampling data we compromise not a lot (given we have multiple models), and gain quite a bit. I am suprised by how effective this was.

        \subsection*{b)}
            \begin{figure}[!h]
                \centering
                \includegraphics[width=\textwidth]{"max_features_5.png"}
                \caption{5 Max Features}
            \end{figure}
            \begin{figure}[!h]
                \centering
                \includegraphics[width=\textwidth]{"max_features_20.png"}
                \caption{20 Max Features}
            \end{figure}
            \begin{figure}[!h]
                \centering
                \includegraphics[width=\textwidth]{"max_features_29.png"}
                \caption{29 Max Features}
            \end{figure}

        \qquad With values under the total features, it seems there is commonly a reduction in correlation. Even with max\_features equal to 29 (one less than all features), there are less-correlated models. 
        Also, as expected, when max\_features returns to the number of features the chart becomes identical with the original chart. (charts on next 2 pages)
        \clearpage
        

    \section*{Q5: Randomness in Clustering}
        \begin{figure}[h]
            \centering
            \includegraphics[width=\textwidth]{"Q5_randomness.png"}
            \caption{SSE of 50 k-means trials (k=5)}
        \end{figure}        
        \qquad This chart is a good example why you should run k-means multiple times. Sometimes, the algorithm doesn't accurately capture the data due to the randomness of initiallization.
        The algorithm finished, and did not make an error, but it still does not represent our data, given the nature of the algorithm.

        \clearpage
    \section*{Q6: Error Vs. K}
        \begin{figure}[!h]
            \centering
            \includegraphics[width=\textwidth]{"Q6_Error_vs_k.png"}
            \caption{SSE of k-means where $k = (1 ... 150)$}
        \end{figure}   
        \qquad First, this computation takes significantly longer as you increase k. 
        Second, it's hard to find what your data actually looks like with just the SSE. We know our data is 3 separate gaussian distributions, but if we choose k by the SSE we should pick higher.
        While the error might be lower, it just means that there are more arbitrary groups on our data and that the \emph{overall} distance from a point to its centroid is small. Not that the centroids accurately represent the data.
    

    \section*{Q7: Clustering Images}
        \subsection*{a)}
        \qquad It seems to be too high. There are primarily three types of images: city, road and nature, with a few outliers.
        \subsection*{b)}
        \qquad I found that $k=4$ was a good k to sort with.
        (Clusters on following pages)
        \begin{figure}[h]
            \centering
            \includegraphics[width=\textwidth]{"k_4_0.png"}
            \caption{Cluster 0}
        \end{figure}  
        \begin{figure}[!h]
            \centering
            \includegraphics[width=\textwidth]{"k_4_1.png"}
            \caption{Cluster 1}
        \end{figure}  
        \begin{figure}[!h]
            \centering
            \includegraphics[width=\textwidth]{"k_4_2.png"}
            \caption{Cluster 2}
        \end{figure}  
        \begin{figure}[!h]
            \centering
            \includegraphics[width=\textwidth]{"k_4_3.png"}
            \caption{Cluster 3}
        \end{figure}  
        \clearpage
        \subsection*{c)}
        \begin{align*}
            \text{SSE of } &k=10 \approx 1200\\
            \text{SSE of } &k=4 \approx 1270
        \end{align*}
        Given that k=4 is a much more sorted group set: No, SSE is not a good indicator.
        
    
    \section*{Q8 Evaluating Clustering as Classification}
    \begin{center}
        \begin{tabular}{c | c c}
            $C_0$ & Nature & 4/50 wrong\\
            $C_1$ & City 1 & 1/50 wrong\\
            $C_1$ & City 2 & 2/50 wrong\\
            $C_1$ & Road & 4/50 wrong
        \end{tabular}
    \end{center}
    \qquad This makes the purity equal to:
    \begin{equation*}
        1-\frac{4+1+2+4}{50*4} = .945
    \end{equation*}
    \section*{Debrief:}
    \noindent1. Approximately 2.5 hrs. on work, 1 hr on report.\\
    2. Easy - I understood k-means well enough from the lectures\\
    3. Alone\\
    4. 95\%\\
    5. I really liked this class, and thank you for having me! :)

\end{document}